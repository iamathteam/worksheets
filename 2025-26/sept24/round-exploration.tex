\documentclass{tufte-handout}

\usepackage{amsmath,amssymb}
\usepackage{enumitem}
\usepackage{graphicx}
\setkeys{Gin}{width=\linewidth,totalheight=\textheight,keepaspectratio}
\graphicspath{{./graphics/}}

\title{Block Practice — Exploration Round: Newton Sums}
\author[IA Math Team]{}
\date{24 September 2025}

\usepackage{booktabs}
\usepackage{units}
\usepackage{fancyvrb}
\fvset{fontsize=\normalsize}

\usepackage{multicol}

\newcommand{\R}{\mathbb{R}}
\newcommand{\nn}[1]{\vspace{#1}\noindent}

\begin{document}

\maketitle

\begin{abstract}
You may approach this either with your team or on your own.
This exploration is entirely \emph{student-centered}: you will \textbf{discover} Newton sums (Newton's identities) by experimenting, forming conjectures, and proving them.

\nn{1em}Enjoy!

\end{abstract}

\section*{Warm-up}
For each polynomial, compute the indicated \textbf{power sums}\footnote{Definition: for a polynomial with roots $r_1,\dots,r_n$ define
\[
P_k=\sum_{i=1}^n r_i^k = r_1^k+r_2^k+\cdots+r_n^k.
\]

Note a special case $P_0=n$
} via Vieta's formulae. 

\vspace{1em}
\begin{enumerate}[label=\textbf{W\arabic*.}, leftmargin=*, itemsep=2em]
  \item Polynomial: $x^2-5x+6$. Compute $P_1, P_2, P_3$ of the roots.
  \item Polynomial: $x^3-6x^2+11x-6$. Compute $P_3$ of the roots.
  \item Polynomial: $x^2-3x+1$. Compute $P_2$ and $P_5$ of the roots.
\end{enumerate}

\vspace{2em}
\section*{Exploration}
\textbf{Goal:} Observe how sums of powers of roots relate to the coefficients of a polynomial.
You will attack this via examples and pattern-finding.

\subsection*{Step A — Observation}
\begin{enumerate}[label=\textbf{A\arabic*.}, leftmargin=*, itemsep=3em]
  \item (Write and align) For the cubic polynomial
  \[
    f(x)=x^3+4x^2+5x+7,
  \]
  let $r_1, r_2,$ and $r_3$ be the roots. Write the three equations $f(r_1)=0,\; f(r_2)=0,\; f(r_3)=0$. \textit{Do not cancel; simply write these equations with $r_i$ substituted for $x$.} To better see the pattern, write one equation beneath the other, aligning like powers (that is, line up the $r_i^3$ terms, the $r_i^2$ terms, the $r_i$ terms, and the constants). 
  \item (Add and observe) Add those three aligned equations term-by-term. What does each column become? (Hint: the column of $r_i^2$ terms becomes $P_2$ times the coefficient, the column of $r_i^1$ terms becomes $P_1$ times the coefficient, and the $r_i^0$ column becomes $P_0$ times the coefficient\footnote{Actually $P_0=\text{power of } f(n)=3$}.)
  \item Let's generalize this! Start with a monic polynomial
\[
f(x)=x^n + c_1 x^{n-1} + c_2 x^{n-2} + \cdots + c_{n-1} x + c_n,
\]

and let its roots be $r_1,\dots,r_n$. Again, each root $r_i$ satisfies $f(r_i)=0$. Repeat the same steps but for this general polynomial: substitute $r_1,r_2,\dots,r_n$ for $x$; you will end up with $n$ equations and a pattern.\footnote{You may want to use dots between the $n$ equations}

\vspace{18em}
Once you are done, you will have the following equation:
\begin{equation}\label{eq:1}
P_n+a_{n-1}P_{n-1}+\cdots+a_1P_1+a_0P_0=0
\end{equation}

Isolate $P_n$. Now, we have written a recursion. However, we do not know the values of $P_{n-1}, P_{n-2},\dots,P_1$ (although $P_0=n$ always). 

\item Find an analog for equation (\ref{eq:1}) for $P_{n-1}, P_{n-2},\dots,P_1,P_0$. Hint: multiply each equation $f(r_i)=0$ by $r_i^{k-n}$.
\vspace{18em}

  % \item (Multiply to see higher powers) To relate higher power sums $P_k$ to coefficients, take the single-root identity $f(r_i)=0$ and multiply it by $r_i^{m}$ for various integers $m\ge0$ (for instance multiply by $r_i$ to get $r_i^{n+1}$ terms, or by $r_i^{k-n}$ to isolate $r_i^k$). After multiplying, repeat the ``write, align, add'' step. What new patterns appear?
  % \item (Record small examples) Using this technique for the cubic above, explicitly produce the identities you get for the sums corresponding to multiplying by $r_i^0$ (no multiplication) and by $r_i^1$. Write the resulting two relations in terms of $S_1,S_2,S_3$ and the coefficients $c_j$.
\end{enumerate}
\vspace{3em}

\subsection*{Step B — Conjecture}
Now turn your observations into a conjecture.

\begin{enumerate}[label=\textbf{B\arabic*.}, leftmargin=*, itemsep=1em]
  \item Based on the observations above, write a conjecture. Now, we can finally get to the question:
  
  \textit{How does $P_k$ relate linearly to previous power sums $P_{k-1},P_{k-2},\dots$ and to the coefficients $c_j$?}

  We caution you this will only work for $k\ge n$.

  \vspace{20em}
  
  \item (Test) Test your conjecture on an easy polynomial where you know the roots (for example $x^3-6x^2+11x-6$ whose roots are $1,2,3$). Check the identities for $k=1,2,3$.

  \vspace{10em}
\end{enumerate}

\subsection*{Step C — Proof}
We leave it now to you to write a concise proof.

\vspace{20em}
\section*{Summary}

Newton sums (also called Newton's identities) produce \emph{power sums} $P_k=\sum r_i^k$ of a polynomial's roots by writing linear relations between $P_k$ and earlier power sums together with the polynomial's coefficients. In contrast, Vieta's formulas directly express the \emph{elementary symmetric sums} (the symmetric polynomials like $r_1+r_2+\cdots$, sum of pairwise products, etc.) in terms of coefficients. Put differently: \textbf{Vieta $\Rightarrow$ elementary symmetric sums; Newton sums $\Rightarrow$ power sums.} The two are related: Newton sums express each power sum as a linear combination whose coefficients are the elementary symmetric sums (equivalently, the polynomial coefficients).

\bigskip
\section*{Practice Problems}
The following are designed to cement the discovery and practice applying Newton sums.

\begin{enumerate}[label=\textbf{P\arabic*.}, leftmargin=*, itemsep=1.2em]
  \item Given $f(x)=x^3-4x^2+5x-2$, compute $P_1,P_2,P_3$ by the aligned-equations method.
  \item For $x^2-3x+1$, use the recurrence you discovered to compute $P_0,P_1,\dots,P_5$ (record the recurrence explicitly).
  \item (Reverse engineering) Suppose a monic cubic has $P_1=2,\; P_2=3,\; P_3=-4$. Use Newton sums to find the polynomial's coefficients $c_1,c_2,c_3$.
  \item (Challenge) Show how Newton sums change if you consider the polynomial whose roots are the reciprocals of the original roots. 
  (Hint: consider $x^n f(1/x)$ and compare power sums of reciprocals.)
  \item (Contest-style) Use Newton sums to compute $r_1^5+r_2^5+r_3^5$ where $r_1,r_2,r_3$ are the roots of $x^3-6x^2+11x-6$.
\end{enumerate}

\bigskip

\end{document}
